Question Answering Systems (QASs) are prominent in making knowledge accessible. Systems such as, for example, in health care (for a survey, see \cite{budler2023review}), have seen rising popularity. Similar attempts have also been made in the legal domain. The COLIEE competition \footnote{\url{https://sites.ualberta.ca/~rabelo/COLIEE2023}} had until recently the Legal Question-Answering challenge, which in the last competitions has been subsumed by Task 3 (Statute Law Retrieval Task) and Task 4 (Legal Textual Entailment Data Corpus). 

Martinez-Gil \cite{martinez2023survey} gives a comprehensive survey of the different approaches and tools and concludes that no solution has managed to provide high accuracy, interpretability, and performance. The reason being that accuracy and inerpretability are mainly obtained by investing considerable human effort, for example those based on ontologies (e.g. \cite{fawei2018methodology}) and linked-data (e.g. \cite{filtz2021linked}).

An important challenge for legal professionals is legal research \cite{sanderson2021practical} due to the ever changing statute and case law. As a consequence of their volume and dynamics, a legal QAS must meet the performance property. Legal QASs in general should also meet the accuracy and interpretability properties. 

The problem addressed in this paper is how the three properties can be obtained within one system, for the purpose of enhancing the legal research process. In order to approach this problem, we first investigate (Sec. \ref{sec:lces}) the needs of legal professionals and characterize the form of Legal QASs that meets their legal research needs. Our main finding is that while legal QASs are useful for professionals, their current methodology involves less looking for answers to specific questions and more looking for positive and negative examples to specific legal concepts within case law. We therefore introduce a new approach to QASs, called Concept-Example Systems (CESs). 

We then follow by surveying the current state-of-the-art in legal QASs with regards to this analysis (Sec. \ref{sec:sota}), where we also argue why our system is better suited for the legal research. The SotA section is followed by an introduction to our methodology (Sec. \ref{sec:method}), as well as a prototype implementation and experiment (Sec. \ref{sec:experiments}). We conclude in the last section.