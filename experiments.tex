\textit{This is a description of the experiments we have conducted -- not the methodology}.

Since this is a preliminary research without a dedicated dataset, we have decided to use
the data from the GDPRHub\footnote{\url{https://gdprhub.eu}} -- a \url{https://noyb.eu} financed site supported by
a large group of volunteers. It uses Wikimedia software to provide easy access and editing features for 
aggregating Data Protection Authorities and national courts' decisions. The software  allows to filter the 
decisions by the GDPR articles relevant for them (usually the articles that were allegedly breached by some 
company or an institution). Besides the text of the decision (automatically translated to English) the database
contains user-created description which contains sections such as: facts, holding, comment and further resources.

In our experiment we have concentrated on the chapter V of GDPR (transfer of the personal data to third countries or 
international organizations), containing articles 44 to 50. We have downloaded all the decisions available for 
these articles and extracted the facts and the holding of their English summary. These sections were sentence-split
using Stanza library \cite{qi2020stanza}.