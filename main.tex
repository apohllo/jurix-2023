\documentclass{IOS-Book-Article}

\usepackage{mathptmx}
\usepackage{url}

%\usepackage{times}
%\normalfont
%\usepackage[T1]{fontenc}
%\usepackage[mtplusscr,mtbold]{mathtime}
%
\begin{document}
\begin{frontmatter}              % The preamble begins here.

%\pretitle{Pretitle}
\title{Giving Examples Instead of Answering Questions: Introducing Legal Concept-Example Systems}
\runningtitle{Legal Concept-Example Systems}
%\subtitle{Subtitle}

\author[A]{\fnms{Tomer} \snm{Libal}%
\thanks{Corresponding Author: Book Production Manager, IOS Press, Nieuwe Hemweg 6B,
1013 BG Amsterdam, The Netherlands; E-mail:
bookproduction@iospress.nl.}},
\author[B]{\fnms{Aleksander} \snm{Smywiński-Pohl}}

\runningauthor{B.P. Manager et al.}
\address[A]{Book Production Department, IOS Press, The Netherlands}
\address[B]{Computer Science Institute, AGH University of Krakow, Poland}

\begin{abstract}
Question-Answering Systems (QASs) have seen a big development in recent years and various attempts have been made to extend them to the legal domain. Nevertheless, the needs and methodology of doctrine legal research relies often less on getting answers and more on finding positive and negative examples for certain legal concepts. In this paper, we introduce a sub-category within QASs that focuses on such legal tasks, motivate its usefulness, design a methodolgy and demonstrate its applicability on a small example.
\end{abstract}

\begin{keyword}
electronic camera-ready manuscript\sep IOS Press\sep
\LaTeX\sep book\sep layout
\end{keyword}
\end{frontmatter}

\thispagestyle{empty}
\pagestyle{empty}

\section{Introduction}

Draft: T

Question Answering Systems (QASs) are prominent in making knowledge accessible. Systems such as [TODO], and [TODO] have been shown to help [TODO]. 


Similar attempts have also been made in the legal domain. The COLIEE competition \footnote{\url{https://sites.ualberta.ca/~rabelo/COLIEE2023}} had until recently the Legal Question-Answering challenge, which in the last competitions has been subsumed by Task 3 (Statute Law Retrieval Task) and Task 4 (Legal Textual Entailment Data Corpus). 

Martinez-Gil \cite{martinez2023survey} gives a comprehensive survey of the different approaches and tools and concludes that no solution has managed to provide high accuracy, interpretability, and performance. The reason being that accuracy and inerpretability are mainly obtained by investing considerable human effort, for example those based on ontologies (e.g. \cite{fawei2018methodology}) and linked-data (e.g. \cite{filtz2021linked}).

An important challenge for legal professionals is doctrinal research \cite{sanderson2021practical} due to the ever changing statute and case law. As a consequence of their volume and dynamics, a legal QAS must meet the performance property. Legal QASs in general should meet the accuracy and interpretability properties. 

The problem addressed in this paper is how the three properties can be obtained within one system. In order to approach this problem, we first investigate (Sec. \ref{sec:lces}) the needs of legal professionals and characterize the form of Legal QASs that meet their legal research needs. Our main finding is that while legal QASs are useful for professionals, their current methodology involves less looking for answers to specific questions and more looking for positive and negative case law examples to specific legal concepts. We therefore introduce a new approach to QASs, called Concept-Example Systems (CESs). 

We then follow by surveying the current state-of-the-art in legal QASs with regards to this analysis (Sec. \ref{sec:sota}) and argue that legal CESs have an advantage over current legal QASs in obtaining the three properties.

We then follow by introducing our methodology and a prototype implementation (Sec. \ref{sec:method}) and conclude in last section.

\section{Legal Concept-Example Systems (LCESs)}
\label{sec:lces}

TODO: T

\subsection{The legal research process}

Before we can present Concept-Example Systems, we would like to understand why they might be useful. We therefore devote the first section to an exposition of the legal research process. For that purpose, we follow the description by Sanderson and Kelly in their ``A practical Guide to Legal Research'' \cite{sanderson2021practical} and take advantage of their running example. 
%It should be noted that while the example takes place in Australia and is based on Australian law, the process in other legal systems does not differ much when considering legal research.
The example is about two Pakistani men, Aban and Salim, who have arrived in Australia with visitor visas and sought refugee status before their visas have expired. They claim they would be prosecuted for being homosexual if they are forced to return to Pakistan and risk being beaten or killed.

Our goal is to build their case for an asylum status in Australia and for that, we need to conduct legal research. 
In their guide, Sanderson and Kelly define the following steps of legal research:

\begin{definition}[Steps of Legal Research \cite{sanderson2021practical}]
The main steps of legal research are:
\begin{enumerate}
    \item identification of the relevant facts;
    \item identification of the legal issues;
    \item identification and interpretation of the rules that govern the legal issues;
    \item application of the rules to the facts; and
    \item conclusion/s.
\end{enumerate}
\end{definition}

Our focus in this paper is step 3. We therefore briefly conclude the findings of the first two steps and ignore steps 4 and 5.

The first step concludes with the following facts - Refugee; Persecution; Homosexual. 
For identifying the legal issues (Step 2), we proceed to identify key legal questions with regards to the facts.

\begin{itemize}
\item Are Aban and Salim refugees? What does the term refugee mean?
\item Can Aban and Salim be persecuted because they are homosexual?
\end{itemize}

Steps 3 guides us in the legal research, which is the main topic of this paper. The key questions to answer in this step are:

\begin{enumerate}
\item What is the relevant legislation? 
\item What cases, if any, have considered similar issues? 
\item Are there relevant international treaties or conventions?
\end{enumerate}

Our paper focuses on legal research about cases, i.e. item 2 above. We therefore omit items 1 and 3. We nevertheless, follow their reasoning and conclude that the relevant legislation is Migration Act 1958 (Cth)\footnote{\url{http://www5.austlii.edu.au/au/legis/cth/consol_act/ma1958118/index.html}}, which references the United Nations Convention Relating to the Status of Refugees\footnote{\url{https://www.unhcr.org/media/convention-and-protocol-relating-status-refugees}}.

From the legislation, we have questions such as:
\begin{itemize}
\item what is a well-founded fear of persecution?
\item is homosexuality a ‘social group’ for the purposes of the Migration Act 1958?
\end{itemize}

The guide states that ``Case law should assist us to determine the meaning of ‘well-founded fear’ and ‘social group'' \cite{sanderson2021practical} and suggests three methods for doing that: (1) by using search terms, (2) by starting from a known similar case, (3) by looking on judicial analysis of a certain legislation. In this paper we focus on the first option.
Modern case law search is mainly done by using special tools, such as the ones offered by LexisNexis and Thomson Reuters Westlaw, or by publicly available indexes, such as EUR-Lex\footnote{\url{http://eur-lex.europa.eu}}.

The main search process includes using search terms and filters in order to locate relevant judgements. If we input in an Australian case law search tool the keywords ``well-founded fear'', together with keywords such as ``homosexuality'' and ``prosecution'', many relevant cases are returned, such as ``S395 v Minister for Immigration and Multicultural Affairs''. 

Once a set of relevant cases was identified, the legal professional needs to read the cases, as summaries or digests are not enough.
A reading of the above mentioned case gives the following paragraph: ``the Tribunal had misdirected itself on the issue of discretion and had not properly considered the appellants’ claims that they had a real and well-founded fear of persecution if they returned to Bangladesh. The real question was whether the harm that would be suffered was such that, {\bf by reason of its intensity or duration, the person cannot reasonably be expected to tolerate it}''.

This paragraph and its concluding remark gives a positive example for a high court decision whether the keyword ``well-founded fear'' holds under the context ``homosexuality'' and ``prosecution'' and under the Australian ``Migration Act 1958''. By using this example, the legal professional can return to her case at hand at determine if the example is similar or not. Clearly, further examples and especially a mix of positive and negative ones would provide the most useful help in determining that.

We argue in this paper that one of the most useful questions for a legal QAS would be to show positive and negative examples of a concept under a certain context. When considering a general purpose legal QAS, it would need to understand the meaning of the question and would not be optimized for finding examples, thus increasing the chance for error, as well as hampering interpretability. It should be noted though, that there are also many cases when the most useful question to ask is not of the above form.

We can now describe a new type of QASs which is tailored for the task above. A methodology for creating such a system, as well as a simple experiment, are described in sections \ref{sec:method} and \ref{sec:experiments}.

\subsection{Concept-Example Systems (CESs)}

A Concept-Example System (CES) differs from a QAS in two aspects:

\begin{itemize}
 \item The starting point for generating a question-answer pair is a legal concept and not a question.
 \item We do not look for specific answers but for as many extracts from court judgements which resolve the concept, under a specific context, either positively or negatively.
\end{itemize}

The reason for using a concept as a starting point is that concepts are, as was seen in the previous sub-section, the basic elements of legal research. Nevertheless, for interpretability, it is important that the user can easily assess the relevancy of the concept and context and we have therefore opted for translating the concept, legislation and context into a question.

%As we will see later in the paper, our questions were generated automatically by using ChatGPT. While manually created questions are not excluded, we believe that automated questions creation is more scalable and erroneous questions can be easily filtered out by the user. We discuss all these points in more depth later.

Once questions are being identified, our goal is to automatically extract from judgement as many paragraphs as possible which can serve as either a positive or a negative example for the concept and under the context defined in the question.

\section{Related work}
\label{sec:sota}

\cite{zhong2020building} is an example of a system that utilizes deep learning for factoid
question answering, following the retrieve-extract paradigm. The authors developed a system
that is aimed at answering questions related to the Chinese building regulations. 
The corpus contains the documents from the Chinese law related to buildings. These documents
are split into passages following the structure of provisions -- each provision is treated 
as a tree and for each non-leaf node a passage containing the content of parent nodes and all 
descendant nodes is created.

The retrieval model is based on a sparse representation of the passages -- the authors use TF-IDF 
\cite{manning1999foundations} vector model for retrieval. Then for top-n returned passages the
apply a Chinese BERT-like model that was trained on the Chinese Wikipedia and the content of the 
regulations. The model is fine-tuned on a collection of question-passage-answer triplets, to 
perform extractive question answering (the answer has to be present literally in the passage).
The training size of the dataset contains 2500 triplets, while the validation and the testing
part contain 500 triplets each.

The authors report very high accuracy of the answers (95\% exact match score for factoid questions 
and 90\% for definitional questions), but this only comprises the answer extraction module. 
The authors do not provide a similar scores for the complete system, but only report the user 
evaluation in terms of easy of use, quality of answers and time spent by the user. The overall performance 
of the system judged by the users is better compared to Baidu search engine and regulation document 
database (however the last system is judged better in terms of accuracy of the answers).


\section{State of the Art}
\label{sec:sota}

TODO: A

\cite{zhong2020building} is an example of a system that utilizes deep learning for factoid
question answering, following the retrieve-extract paradigm. The authors developed a system
that is aimed at answering questions related to the Chinese building regulations. 
The corpus contains the documents from the Chinese law related to buildings. These documents
are split into passages following the structure of provisions -- each provision is treated 
as a tree and for each non-leaf node a passage containing the content of parent nodes and all 
descendant nodes is created.

The retrieval model is based on a sparse representation of the passages -- the authors use TF-IDF 
\cite{manning1999foundations} vector model for retrieval. Then for top-n returned passages the
apply a Chinese BERT-like model that was trained on the Chinese Wikipedia and the content of the 
regulations. The model is fine-tuned on a collection of question-passage-answer triplets, to 
perform extractive question answering (the answer has to be present literally in the passage).
The training size of the dataset contains 2500 triplets, while the validation and the testing
part contain 500 triplets each.

The authors report very high accuracy of the answers (95\% exact match score for factoid questions 
and 90\% for definitional questions), but this only comprises the answer extraction module. 
The authors do not provide a similar scores for the complete system, but only report the user 
evaluation in terms of easy of use, quality of answers and time spent by the user. The overall performance 
of the system judged by the users is better compared to Baidu search engine and regulation document 
database (however the last system is judged better in terms of accuracy of the answers).

\section{Methodology}
\label{sec:method}

TODO: A

As it was discussed in Section \ref{sec:sota} there are two dominant approaches 
based on neural networks that are used to solve the problem of QA:
\begin{enumerate}
  \item retrieve and extract/generate answer, e.g. , %RAG
  \item generate an answer using a large language model (LLM). %GPT-3
\end{enumerate}
Our approach -- LCES -- differs from them, since:
\begin{enumerate}
  \item we do not require that the user poses a question to the system, since the system might generate a question for the user,
    taking into account the legislation,
  \item we do not answer the question, but give positive and negative examples, taken directly from the relevant court decisions.
\end{enumerate}
Our approach is most similar to the retrieve and generate paradigm, since when searching for relevant information we use
a cross-encoder (typically used to re-score the results of a sparse or a dense retriever). Yet we also introduce an important
difference in that respect, by taking individual sentences as the primary means for bearing atomic pieces of information.  
As a result we resolve the problem of hallucination present in the second mentioned approach, i.e. the approach based on LLMs.

The procedure of finding the relevant examples is divided into the following steps:
\begin{enumerate}
  \item Given a legal concept, generate a question that asks if the concept was applicable in a given context,
    using a first-order language\footnote{The first-order language is used in the Tarski's sense -- to differentiate
    between the first-order language and the meta-language that is used to speak about the first-order language. In this 
    context it means that the questions contains the legal concept as a normal lexical unit, rather than a quoted phrase.
    For example, for the concept \textit{data subject rights} a question in first-order language would be 
      ,,Were the data subjects informed about their rights regarding their personal data?'', vs. 
      a meta-language level question ,,Does the passage contain a $\ll$data subject rights$\gg$ phrase?''.}.
    For example, the term \textit{legally binding and enforceable instrument} generates the following question:
    \textit{Does the instrument in question meet the criteria for being legally binding under relevant laws?}
  \item Having the question generated, we use a binary classification model to find sentences in the court cases that answer the question. 
    We use the predicted probability not only to sort these sentences, but also to filter sentences below a defined threshold.
    As a result this step might generate an empty set, meaning that in our corpus there are no relevant court decisions. 
    This prevents the system from presenting passages that are irrelevant, but also limits the possibility of false
    classification of the sentences.
  \item In the last step we extend the sentence to include the preceding and the following sentence forming a three-sentence passage
    and ask an instruction-following model to judge if the example gives a positive or a negative answer to the generated question.
    We use that prediction to classify the passage as a positive or a negative example.
\end{enumerate}

It has to be stressed that none of the steps uses a domain-specific (legal) question-answering dataset. We only use an open-domain
dataset (SQuAD) and general generative models (ChatGPT and Flan-T5). This is a very important feature of our approach,
since the oder neural systems related to legal question answering are typically trained on legal datasets.
The detailed description of the procedure is given in the following sections.

\subsection{Question generation}

\textit{Tomer: provide the rationale for the concept selection and the prompt for ChatGPT used to generate the questions.}

We use ChatGPT to generate the questions using the legal concepts as the input. For each legal concept we have generated
4 specific and 1 general question. In general the number of questions is a parameter of the approach and a higher number
should yield a better recall of the results. So far we have left the influence of that parameter on the result as well
as a selection of particular language model for future research.

Since we test our approach on GDPR, we assume that ChatGPT has a good knowledge
of the relevant legislation, since it should be present in the corpus, the
system was trained on. As the results show (cf. Table
\ref{tab:question-examples}), this is a valid assumption, however for a solution
that is independent from the knowledge of the language model (i.e. taking into
account some amendments that were applied recently), we could provide the
relevant provision as a part of the prompt.

\begin{table}[htbp]
  \centering\begin{tabular}{l p{6cm}}
    \hline
    \textbf{Legal concept} & \textbf{Generated question} \\
    \hline
Enforceable data subject rights & Were the data subjects informed about their rights regarding their personal data?\\
\hline
Effective legal remedies for data subjects & Are there established procedures within the organization for data subjects to seek legal remedies?\\
\hline
Legally binding and enforceable instrument & Does the instrument in question meet the criteria for being legally binding under relevant laws?\\
\hline
Binding corporate rules & Does the organization have binding corporate rules in place for data protection?\\
\hline
Standard data protection clauses & Has the organization implemented Standard Contractual Clauses (SCCs) for data transfers?\\
\hline
Approved code of conduct & Has the organization's code of conduct been officially approved by the relevant authority?\\
\hline
Approved certification mechanism & Has the organization's certification mechanism been officially approved by the relevant authority?\\
\hline
Competent supervisory authority & Has the organization received approvals or guidance from the competent supervisory authority for its data protection practices?\\
\hline
Contractual clauses & Does the organization incorporate specific data protection clauses in its contracts with third parties?\\
\hline
Administrative arrangements & Does the organization have administrative arrangements in place that include provisions for data subject rights?\\
\hline
Consistency mechanism & Does the organization employ a consistency mechanism to ensure uniform data protection practices?\\
\hline

  \end{tabular}
  \caption{Example questions generated by ChatGPT for the legal concepts taken from GDPR.}
  \label{tab:question-examples}
\end{table}

We argue that this application for generative AI in the legal domain is justified and interpretable. 
Generative AI is very good at rephrasing sentences and expanding concepts (as the results show clearly). 
By applying this method we remove the burden of generating similar question from the user, 
so the proposed approach preserves time of the user. On the other hand, since the generated questions 
might be presented to the user, the process is interpretable at the human level \cite{martinez2023survey}, meaning that
even though the workings of the neural network cannot be tracked, its outcome can be understood by humans and 
possible errors might be identified as wrongly generated questions. Once again removal of such invalid questions 
is a much faster process than their invention by the potential users of the solution.

\subsection{Sentence retrieval}

The next step in the procedure is the retrieval of the relevant pieces of judgments that include answers to the
generated questions. As explained in the Section \ref{sec:experiments}, since the collection of text we use in our experiments
is small, compared to other QA dataset, we have decided to use a simplified approach for retrieving the relevant
passages of the decisions. Instead of first retrieving a subset of the data using a sparse or a dense retriever,
we have implemented only a cross-encoder, which is typically used for re-scoring the results on a subset of documents.
In our approach the cross-encoder is run for each generated question and each available passage. 

As the cross-encoder we have used a binary classification model on the top of Albert \cite{lan2019albert} model (from the BERT family
\cite{devlin2018bert} of encoder-only transformer models). We have used SQuAD dataset \cite{rajpurkar2016squad},
specifically the second version with unanswerable questions \cite{rajpurkar2018know}. We have implemented
an indirect method of contrastive learning (i.e. the contrastive goal is not encoded in the loss but in the preparation of
the  training examples). A typical procedure for SQuAD 2.0, to build a binary classifier judging if the passage 
contains an answer to the question would be to use the unanswerable questions together with the passages as the 
negative examples and the answerable questions with their passages as the positive examples. Yet, that would yield
a highly imbalanced dataset, which does not reflect the distribution of positive and negative question -- passage 
pairs of the application domain (in SQuAD 2.0 there are 20\% of unanswerable question -- passage pairs, while 
for the retrieval more than 99\% of passages are irrelevant). 

To resolve that problem we have pre-processed SQuAD 2.0 by splitting each passage into individual sentences and identified
those sentences that included the whole answer. This is straightforward operation, since SQuAD implements the extractive
QA paradigm which requires that the answer is directly present in the passage as a sequence of consecutive tokens,
so the sentence that includes those tokens together with the question is taken as a positive example, while
all the other sentences from the same passage and that question are treated as the negative examples.
We call this approach an indirect contrastive learning, since the negative examples will include information 
which is highly relevant for the question, but they do not include the actual answer. For the unanswerable questions
we also split the passage into individual sentences and treat all of them as negative examples. As a result we 
obtain a dataset that includes a high number of negative examples, which are still highly relevant with respect to the 
question.

\subsection{Passage classification}


\section{Experiments}
\label{sec:experiments}

Since this is a preliminary research without a dedicated dataset, we have decided to use the data from the GDPRHub\footnote{\url{https://gdprhub.eu}} -- a \url{https://noyb.eu} financed site supported by
a large group of volunteers. It uses Wikimedia software to provide easy access and editing features for 
aggregating Data Protection Authorities and national courts' decisions. The software  allows to filter the 
decisions by the GDPR articles relevant for them (usually the articles that were allegedly breached by some 
company or an institution). Besides the text of the decision (automatically translated to English) the database
contains user-created description which contains sections such as: facts, holding, comment and further resources.

In our experiment we have concentrated on the chapter V of GDPR (transfer of the personal data to third countries or 
international organizations), concentrating on articles 44 to 46. We have downloaded all the decisions available for 
these articles (81 decisions in total, 47 unique decisions) and extracted the facts and the holding of their English 
summary. These sections were sentence-split using the Stanza library \cite{qi2020stanza}, yielding 1201 sentences
of which 1140 were unique. 

By analyzing the selected GDRP articles, we have selected 11 legal concepts that were the most relevant for them:
\textit{Enforceable data subject rights}, \textit{Effective legal remedies for data subjects}, 
\textit{Legally binding and enforceable instrument},
\textit{Binding corporate rules}, \textit{Standard data protection clauses}, \textit{Approved code of conduct},
\textit{Approved certification mechanism}, \textit{Competent supervisory authority}, \textit{Contractual clauses},
\textit{Administrative arrangements}, \textit{Consistency mechanism}. For each concept we have asked ChatGPT 
to generate 4 specific and one general question related to the relevant legislation. Some examples of these
questions are given in Table \ref{tab:question-examples}.

For training the sentence retrieval module we have processed all SQuAD v.2 \cite{rajpurkar2018know} contexts applying Stanza 
library \cite{qi2020stanza}. We have followed the original split of the dataset into training and validation parts.
The procedure gave more than 650 thousand training examples and more than 63 thousand validation examples. There were 87 thousand
positive examples in the training set (13\%) and almost 6 thousand positive examples in the validation set (9.5\%).
We have fine-tuned two models on the data: BERT large uncased \cite{devlin2018bert} (\texttt{bert-large}) and 
Albert XXL v.1 \cite{lan2019albert} (\texttt{albert-xxl}). The results of the training are given in Table \ref{tab:bert-albert}.
The reported time of training is for 4 instances of A100 40GB NVIDIA GPU.

\begin{table}[htbp]
    \centering\begin{tabular}{l r r r r}
        \hline
        \textbf{Model} & \textbf{Accuracy} & \textbf{F1-score} & \textbf{Steps} & \textbf{Time} \\
        \hline
        \texttt{bert-large} & 95.77 & 78.15 & 50 000 & 20h \\
        \hline
        \texttt{albert-xxl} & \textbf{97.05} & \textbf{84.14} & 5 000 & 24h \\
        \hline
    \end{tabular}
    \caption{The results of the training BERT large uncased and Albert XXL on the sentence-split SQuAD dataset.}
    \label{tab:bert-albert}
\end{table}

On the validation set \texttt{bert-large} gave results much worse than \texttt{albert-xxl}, even though it was trained 10-times longer 
(in terms of the number of steps, not the wall clock time; the total training time for Albert is 4h longer). 
As a result we have decided to use the fine-tuned \texttt{albert-xxl} 
in the following experiments. To check if the model is behaving according to our expectation, we have asked several questions,
to see if the results are meaningful. For instance, for the question: \textit{Does licensing software for processing personal data 
makes the company a data processor?} the model found a sentence: \textit{The EU subsidiary of the American corporation Microsoft, 
established in Ireland, has access to personal data on the Hub as it licenses the software necessary to operate it (data processor).}
and assigned 78.38\% probability, that this sentence contains the answer to the question.

Having the dataset of sentences from GDPRHub, 55 questions generated by ChatGPT and the classification model described above,
we have applied the model on the questions and all sentences from the GDPRHub. For each question we have manually inspected top-5 results
produced by the classification model, even for cases when the model yielded probability below 0.5, in order to see if the default
decision threshold works as expected. As a result we were able to identify 0.65 as the optimal (equal error rate) threshold for our 
experiment. There were 18 such sentences (with 2 duplicates, obtained for similar questions) with 4 false positives (77.78\% precision).
There were also 4 false negatives (identified among the top-5 results below 0.65 threshold), giving 77.78\% recall and the same F1-score. 
The results obtained by our model are on par with the state-of-the-art neural passage retrieval model -- DPR \cite{karpukhin2020dense},
which obtains 72.24 accuracy on top-5 results tested on the Natural Questions dataset \cite{kwiatkowski2019natural}. Obviously, 
our results should be treated with a grain of salt, since we were biased towards positive assessment of our approach and the 
datasets is orders of magnitude smaller, yet they still show that the approach produces meaningful preliminary results.

As the last step we have provided all results with probability above 0.65 and the 4 remaining positive results below that score
to the Flan-T5-large model \cite{wei2021finetuned,raffel2020exploring}, to obtain an answer if the example is positive or negative
with respect to the given question. The sentences were extended with one preceding and one following sentence, to form a richer context.
We have manually reviewed the answers provided by the model. Out of 32 analyzed examples (including those  
that were previously judged invalid), 18 (\textbf{56.25\%}) were correct (this result included 2 duplicates) and 14 were incorrect (this set
also included 2 duplicates). 

As an example for the concept: \textit{Legally binding and enforceable
instrument} a following question was generated: \textit{Does the instrument in
question meet the criteria for being legally binding under relevant laws?}
In the GDPRHub corpus the model found a sentence, which after an expansion formed
the following example: \textit{In its privacy policy the controller simply
informed consumers about data transfers to third parties and countries, without
any further legal effect.  \textbf{This document did not constitute a legally binding
contract offered to customers by the controller, as the Consumer Center
suggested.} The court also held that the Consumer Center's claim with regard to
the cookie banners was unfounded.} The Flan T5 model judged simply that the
answer is \textit{no}, so this context would be presented to the user as a
negative example.

TODO Tomer: write that the above experiment is for sanity check and that a comprehensive experiment will take place later (with Tereza?)

\section{Conclusion}

TODO: last

In this paper we introduced a sub-system of QAS and discussed its merits and a possible methodology for its creation. We also demonstrated via a small experiment how the methodology can be implemented and what kind of results it might return.

The work introduced in this paper is an initial work towards high-quality CESs creation. As such, there are numerous ways to optimize the process, which we consider as future work. Due to lack of space, we mention but a few of them next.

Questions are generated using LLMs fine-tuned on the same legal datasets which are used to identify examples.




\bibliographystyle{vancouver}
\bibliography{main}

%\bibitem{filtz21}
%Filtz, E. et al. "The linked legal data landscape: linking legal data across different countries." \textit{Artificial Intelligence and Law 29.4} (2021): 485-539.
%
%\bibitem{kelly21}
%Sanderson, J. et al. A Practical Guide to Legal Research. \textit{Lawbook Co.}, (2021).
%
%\bibitem{martinez23}
%Martinez-Gil, J. "A survey on legal question–answering systems." \textit{Computer Science Review 48} (2023): 100552.
%
%\bibitem{wyner18}
%Fawei, B. et al. A Methodology for a Criminal Law and Procedure Ontology for Legal Question Answering. \textit{ Semantic Technology. JIST 2018. Lecture Notes in Computer Science}, vol 11341. Springer, Cham. (2018)

\end{document}
